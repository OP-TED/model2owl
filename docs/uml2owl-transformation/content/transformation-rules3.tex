%\twocolumn
\section{Transformation of UML datatypes}
\label{sec:tran-rules3}

In this section are specified transformation rules for UML datatypes and enumerations. Table \ref{tab:datatype-overview} provides an overview of the section coverage.

\begin{table}[!h]
\centering
\begin{tabular}{@{}p{0.3\textwidth}p{0.2\textwidth}p{0.2\textwidth}p{0.2\textwidth}@{}}
\toprule
UML element            & Rules \coreComponent & Rules \shaclComponent & Rules \reasoningComponent \\ \midrule
Primitive datatype & Rule \ref{rule:datatype-core} &  &  \\
Structured datatype & Rule \ref{rule:datatype-structured-core} &  &  \\
Enumeration & Rule \ref{rule:enumeration-core} &  & Rule \ref{rule:enumeration-rc} \\
Enumeration item & Rule \ref{rule:enumeration-item-core} &  &  \\ \bottomrule
\end{tabular}
\caption{Overview of transformation rules for UML datatypes}
\label{tab:datatype-overview}
\end{table}

\subsection{Primitive datatype}
\label{sec:primitive-type}

The UML primitive type defines a predefined datatype without any
substructure. The UML specification \citep{uml2.5} predefines five primitive types: String,
Integer, Boolean, UnlimitedNatural and Real.
Here we extended those to the list provided in Table \ref{tab:type-mapping}.

\begin{figure}[!ht]
	\centering
	\begin{subfigure}{.5\textwidth}
		\centering
		\begin{tikzpicture}
			\umlsimpleclass[type=DataType]{Text}
			\umlsimpleclass[type=DataType,right=2em of Text]{xsd{:}boolean}
			\umlsimpleclass[type=DataType,below=1em of xsd{:}boolean]{DatatypeName}
		\end{tikzpicture}		
	\end{subfigure}%
	\begin{subfigure}{.5\textwidth}
		\centering
		\begin{tikzpicture}
			\gDatatype{}{s}{xsd:string}
			\gDatatype{right=2em of s}{b}{xsd:boolean}
			\gDatatype{below=2em of b}{dtn}{:DatatypeName}
		\end{tikzpicture}
	\end{subfigure}
	\caption{Visual representation of an UML Datatype (on the left) and an OWL Datatype (on the right)}
	\label{fig:datatype-visual}
\end{figure}

\begin{trule}[Datatype -- \coreComponent]
	\label{rule:datatype-core}
	Specify datatype declaration axiom for UML datatype as follows:
	\begin{itemize}
		\item UML primitive datatypes are declared as the mapped XSD datatype in Table \ref{tab:type-mapping}.
		\item XSD and RDF(S) datatypes are declared as such.
		\item Model specific datatypes are declared as such.
	\end{itemize}
\end{trule}

\begin{table}[!ht]
\centering
\begin{tabular}{@{}ll@{}}
\toprule
UML datatype & XSD datatype \\ \midrule
Boolean & xsd:boolean \\
Float & xsd:float \\
Integer & xsd:integer \\
Char & xsd:string \\
String & xsd:string \\
Short & xsd:short \\
Long & xsd:long \\
Decimal & xsd:decimal \\
Real & xsd:float \\
Date & xsd:date \\
Numeric & xsd:integer \\
Text & xsd:string \\ \bottomrule
\end{tabular}
\caption{Mapping of UML primitive types to XSD datatypes}
\label{tab:type-mapping}
\end{table}

\vspace{-\parskip}
\begin{minipage}[b]{.394\textwidth}
\begin{lstlisting}[language=Turtle, caption={Datatype declaration in Turtle syntax}, captionpos=b]
xsd:string a rdfs:Datatype ;
  rdfs:label "String"@en ;
  skos:definition "Description of the datatype meaning"@en ;
.
xsd:boolean a rdfs:Datatype ;
  rdfs:label "Boolean"@en ;
  skos:definition "Description of the datatype meaning"@en ;
.
:DatatypeName a rdfs:Datatype ;
  rdfs:label "Datatype name"@en ;
  skos:definition "Description of the datatype meaning"@en ;
.
\end{lstlisting}
\end{minipage}% <---------------- Note the use of "%"
\quad\vspace{-\parskip}
\begin{minipage}[b]{.62\textwidth}
%\raggedleft
\begin{lstlisting}[language=RDF/XML, caption={Datatype declaration in RDF/XML syntax}, captionpos=b]
<rdfs:Datatype rdf:about="http://www.w3.org/2001/XMLSchema#string">
  <rdfs:label xml:lang="en">String</rdfs:label>
  <skos:definition xml:lang="en">Description of the datatype meaning</skos:definition>  
</rdfs:Datatype>
<rdfs:Datatype rdf:about="http://www.w3.org/2001/XMLSchema#boolean">
  <rdfs:label xml:lang="en">Boolean</rdfs:label>
  <skos:definition xml:lang="en">Description of the datatype meaning</skos:definition>  
</rdfs:Datatype>
<rdfs:Datatype rdf:about="http://base.uri/DatatypeName">
  <rdfs:label xml:lang="en">Datatype name</rdfs:label>
  <skos:definition xml:lang="en">Description of the datatype meaning</skos:definition>  
</rdfs:Datatype>
\end{lstlisting}
\end{minipage}
\vspace{-\parskip}

\subsection{Structured datatypes}

The UML structured datatype \citep{uml2.5} has attributes and is used to define complex data types. The structured datatypes should be treated as classes.

\begin{trule}[Structured Datatype -- \coreComponent]
	\label{rule:datatype-structured-core}	
	Specify OWL class declaration axiom for UML structured datatype.
\end{trule}

\subsection{Enumeration}

UML Enumerations \citep{uml2.5} are kinds of datatypes, whose values correspond to one
of user-defined literals. They should be transformed into SKOS \citep{skos-spec} concept schemes comprising the concepts corresponding to enumerated items. 

\begin{figure}[!ht]
	\centering
	\begin{tikzpicture}
		\umlclass[type=Enumeration]{ControlledList}{
		itemA='First item'\\
		itemB='Second item'}{}
	\end{tikzpicture}		
	\caption{Visual representation of an UML Enumeration}
	\label{fig:enumeration-uml-visual}
\end{figure}

\begin{figure}[!ht]
	\centering
	\begin{tikzpicture}
		\gClass{}{cs}{skos:ConceptScheme}
		\gClass{right=6em of cs}{c}{skos:Concept}
		
		\gInstance{below=4em of cs.south}{left}{icl}{:ControlledList}
		\gInstance{below=3.5em of c.south east}{right}{ia}{:ItemA}
		\gInstance{below=7em of c.south west, xshift=-2em}{right}{ib}{:ItemB}
		
		\gLiteral{below=4em of icl}{icl-l}{\valueLang[en]{Controlled list}}
		\gLiteral{below=4em of ia}{ia-l}{\valueLang[en]{Item A}}
		\gLiteral{below=4em of ib}{ib-l}{\valueLang[en]{Item B}}
		
		\gPredicate{icl}{cs}{rdf:type}	
		\gPredicate{ia}{c}{rdf:type}
		\gObjectProperty{ib}{c}{$\dots$}
		
		\gAnnotationProperty{icl}{icl-l}{skos:prefLabel}
		\gAnnotationProperty{ia}{ia-l}{skos:prefLabel}
		\gAnnotationProperty{ib}{ib-l}{skos:prefLabel}
		
		\gObjectProperty{ia}{icl}{skos:inScheme}
		\gObjectProperty{ib}{icl}{$\dots$}
		
	\end{tikzpicture}		
	\caption{Visual representation of a SKOS concept scheme with concepts}
	\label{fig:enumeration-owl-visual}
\end{figure}

\begin{trule}[Enumeration -- \coreComponent]
	\label{rule:enumeration-core}
	Specify SKOS concept scheme instantiation axiom for an UML enumeration.
\end{trule}

\vspace{-\parskip}
\begin{minipage}[b]{.4\textwidth}
\begin{lstlisting}[language=Turtle, caption={Concept scheme instantiation in Turtle syntax}, captionpos=b]
:ControlledList a skos:ConceptScheme ;
  rdfs:label "Controlled list" ;
  skos:prefLabel "Controlled list"@en ;
  skos:definition "Definition of the concept scheme meaning"@en ;
.
\end{lstlisting}
\end{minipage}% <---------------- Note the use of "%"
\quad\vspace{-\parskip}
\begin{minipage}[b]{.6\textwidth}
%\raggedleft
\begin{lstlisting}[language=RDF/XML, caption={Concept scheme instantiation in RDF/XML syntax}, captionpos=b]
<skos:ConceptScheme rdf:about="http://base.uri/ControlledList">
  <rdfs:label>Controlled list</rdfs:label>
  <skos:prefLabel xml:lang="en">Controlled list</skos:prefLabel>
  <skos:definition xml:lang="en">Definition of the concept scheme meaning</skos:definition>
</skos:ConceptScheme>
\end{lstlisting}
\end{minipage}
\vspace{-\parskip}

\enlargethispage{2em}
\begin{trule}[Enumeration items -- \coreComponent]
	\label{rule:enumeration-item-core}
	Specify SKOS concept instantiation axiom for an UML enumeration item.
\end{trule}

\vspace{-\parskip}
\begin{minipage}[b]{.394\textwidth}
\begin{lstlisting}[language=Turtle, caption={Concept instantiation in Turtle syntax}, captionpos=b]
:itemA a skos:Concept ;
  skos:inScheme :ControlledList ;
  rdfs:label "Item A" ;
  skos:prefLabel "Item A"@en ;
  skos:definition "Description fo the concept meaning"@en ;
.
:itemB a skos:Concept ;
  skos:inScheme :ControlledList ;
  rdfs:label "Item B" ;
  skos:prefLabel "Item B"@en ;
  skos:definition "Description fo the concept meaning"@en ;
.
\end{lstlisting}
\end{minipage}% <---------------- Note the use of "%"
\quad\vspace{-\parskip}
\begin{minipage}[b]{.62\textwidth}
%\raggedleft
\begin{lstlisting}[language=RDF/XML, caption={Concept instantiation in RDF/XML syntax}, captionpos=b]
<skos:Concept rdf:about="http://base.uri/itemA">
  <skos:inScheme rdf:resource="http://base.uri/ControlledList"/>
  <rdfs:label>Item A</rdfs:label>
  <skos:prefLabel xml:lang="en">Item A</skos:prefLabel>
  <skos:definition xml:lang="en">Description fo the concept meaning</skos:definition>
</skos:Concept>
<skos:Concept rdf:about="http://base.uri/itemB">
  <skos:inScheme rdf:resource="http://base.uri/ControlledList"/>
  <rdfs:label>Item B</rdfs:label>
  <skos:prefLabel xml:lang="en">Item B</skos:prefLabel>
  <skos:definition xml:lang="en">Description fo the concept meaning</skos:definition>
</skos:Concept>
\end{lstlisting}
\end{minipage}
\vspace{-\parskip}

\begin{trule}[Enumeration -- \reasoningComponent]
	\label{rule:enumeration-rc}
	For an UML enumeration,	specify an equivalent class restriction covering the set of individuals that are \texttt{skos:inScheme} of this enumeration.
\end{trule}

\vspace{-\parskip}
\begin{minipage}[b]{.394\textwidth}
\begin{lstlisting}[language=Turtle, caption={In-scheme equivalent class in Turtle syntax}, captionpos=b]
:ControlledList a owl:Class ;  
  owl:equivalentClass [
    rdf:type owl:Restriction ;
    owl:allValuesFrom :ControlledList ;
    owl:onProperty skos:inScheme ;
  ] ;  
.
\end{lstlisting}
\end{minipage}% <---------------- Note the use of "%"
\quad\vspace{-\parskip}
\begin{minipage}[b]{.62\textwidth}
%\raggedleft
\begin{lstlisting}[language=RDF/XML, caption={In-scheme equivalent class in RDF/XML syntax}, captionpos=b]
<owl:Class rdf:about="http://base.uri/ControlledList">  
  <owl:equivalentClass>
    <owl:Restriction>
      <owl:onProperty rdf:resource=".../02/skos/core#inScheme"/>
      <owl:hasValue rdf:resource="http://base.uri/ControlledList"/>
    </owl:Restriction>
  </owl:equivalentClass>
</owl:Class>
\end{lstlisting}
\end{minipage}
\vspace{-\parskip}