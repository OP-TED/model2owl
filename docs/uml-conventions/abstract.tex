\section*{Abstract}

	In the eProcurement ontology initiative, the conceptual model is represented in Unified Modelling Language (UML). It is a visual representation language that facilitates understanding and convergence between stakeholders towards a common conceptualisation of the model.
	
	UML does not define a formal semantics that would permit to determine, from the class diagrams, whether an ontology is consistent; or to determine the correctness of the ontology implementation. Semantics in such cases becomes a subject to interpretation by the stakeholders involved in the development process and later by the users in the application and integration tasks.
	
	On the other hand, UML is closer than more logic-oriented approaches to the programming languages in which enterprise applications are implemented. The UML Conceptual Model of the eProcurement domain serves as the single source of truth, which means that the formal eProcurement ontology is derived from it through a model transformation process.
	
	For this reason, the current specification establishes conventions for interpretation of the UML-based conceptual model. It provides the UML modelling constraints and a set of conventional and technical recommendations for naming and structuring the UML class diagrams.
	
	