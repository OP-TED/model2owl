\section{Transformation rules}
\label{sec:tran-rules}


\subsection{Classes}
\label{sec:classes}

UML represents atomic classes as named elements of type \textit{Class} without further features. In OWL, the atomic class, \texttt{owl:Class}, has no intension. It can only be interpreted by its name that has a meaning in the world outside the ontology. The atomic class is a class description that is simultaneously a class axiom. [Kiko2008]



\begin{figure}[!ht]
	\centering
	\begin{subfigure}{.5\textwidth}
		\centering
		\includegraphics[width=.45\linewidth]{images/background.png}
		\caption{UML class}
		\label{fig:class-uml}
	\end{subfigure}%
	\begin{subfigure}{.5\textwidth}
		\centering
		\includegraphics[width=.45\linewidth]{images/background.png}
		\caption{OWL class}
		\label{fig:class-owl}
	\end{subfigure}
	\caption{Visual representation in UML and OWL of a prototypical class }
	\label{fig:class}
\end{figure}

\begin{lstlisting}[caption={OWL Class},captionpos=b]
Class: C
 skos:prefLabel "label"[en]
 skos:definition "definition"
\end{lstlisting}

\begin{trule}[Class]
	\label{rule:class}
	Given an UML element $E$ of type $Class$ with a name and optionally with a description  create an OWL class $C$. 
	\begin{itemize}
		\item The name is transformed into URI and the label of $C$.
		\item The description of $E$ is transformed into 
	\end{itemize}		 	
\end{trule}

\begin{figure}
	\centering
	\begin{tikzpicture}
	\begin{umlpackage}{p}
	\begin{umlpackage}{sp1}
	\umlclass[template=T]{A}{
		n : uint \\ t : float
	}{}
	\umlclass[y=-3]{B}{
		d : double
	}{
		\umlvirt{setB(b : B) : void} \\ getB() : B}
	\end{umlpackage}
	\begin{umlpackage}[x=10,y=-6]{sp2}
	\umlinterface{C}{
		n : uint \\ s : string
	}{}
	\end{umlpackage}
	\umlclass[x=2,y=-10]{D}{
		n : uint
	}{}
	\end{umlpackage}
	
	\umlassoc[geometry=-|-, arg1=tata, mult1=*, pos1=0.3, arg2=toto, mult2=1, pos2=2.9, align2=left]{C}{B}
	\umlunicompo[geometry=-|, arg=titi, mult=*, pos=1.7, stereo=vector]{D}{C}
	\umlimport[geometry=|-, anchors=90 and 50, name=import]{sp2}{sp1}
	\umlaggreg[arg=tutu, mult=1, pos=0.8, angle1=30, angle2=60, loopsize=2cm]{D}{D}
	\umlinherit[geometry=-|]{D}{B}
	\umlnote[x=2.5,y=-6, width=3cm]{B}{Je suis une note qui concerne la classe B}
	\umlnote[x=7.5,y=-2]{import-2}{Je suis une note qui concerne la relation d'import}
	\end{tikzpicture}
	\caption{This is a UML diagram}
	\label{fig:diagram1}
\end{figure}

\begin{figure}
	\centering
	\begin{tikzpicture}
	\begin{umlpackage}{p}
	\begin{umlpackage}{sp1}
	\umlclass[template=T]{A}{
		n : uint \\ t : float
	}{}
	\umlclass[y=-3]{B}{
		d : double
	}{
		\umlvirt{setB(b : B) : void} \\ getB() : B}
	\end{umlpackage}
	\begin{umlpackage}[x=10,y=-6]{sp2}
	\umlinterface{C}{
		n : uint \\ s : string
	}{}
	\end{umlpackage}
	\umlclass[x=2,y=-10]{D}{
		n : uint
	}{}
	\end{umlpackage}
	
	\umlassoc[geometry=-|-, arg1=tata, mult1=*, pos1=0.3, arg2=toto, mult2=1, pos2=2.9, align2=left]{C}{B}
	\umlunicompo[geometry=-|, arg=titi, mult=*, pos=1.7, stereo=vector]{D}{C}
	\umlimport[geometry=|-, anchors=90 and 50, name=import]{sp2}{sp1}
	\umlaggreg[arg=tutu, mult=1, pos=0.8, angle1=30, angle2=60, loopsize=2cm]{D}{D}
	\umlinherit[geometry=-|]{D}{B}
	\umlnote[x=2.5,y=-6, width=3cm]{B}{Je suis une note qui concerne la classe B}
	\umlnote[x=7.5,y=-2]{import-2}{Je suis une note qui concerne la relation d'import}
	\end{tikzpicture}
	\caption{This is a UML diagram}
	\label{fig:diagram2}
\end{figure}

\subsection{Generalisation}
\label{sec:generalisation}
...




