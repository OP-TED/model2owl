%\twocolumn
\section{Transformation of UML classes and attributes}
\label{sec:tran-rules}

In this section are specified transformation rules for UML class and attribute elements. Table \ref{tab:class-attribute-overview} provides an overview of the section coverage.

\begin{table}[!h]
\centering
\begin{tabular}{@{}p{0.3\textwidth}p{0.2\textwidth}p{0.2\textwidth}p{0.2\textwidth}@{}}
\toprule
UML element            & Rules \coreComponent & Rules \shaclComponent & Rules \reasoningComponent \\ \midrule
Class                  & Rule \ref{rule:class-core}    & Rule \ref{rule:class-ds}  &       \\
Abstract class         &      & Rule \ref{rule:class-abstract-ds}  &       \\
Attribute              & Rule \ref{rule:attribute-core}    &   &  Rule \ref{rule:attribute-rc-domain}     \\
Attribute type         &      & Rule \ref{rule:attribute-ds-range}  & Rule \ref{rule:attribute-rc-range}     \\
Attribute multiplicity &      & Rule \ref{rule:attribute-ds-multiplicity}  & Rule \ref{rule:attribute-rc-multiplicity}, \ref{rule:attribute-rc-multiplicity-one}     \\ \bottomrule
\end{tabular}
\caption{Overview of transformation rules for UML classes and attributes}
\label{tab:class-attribute-overview}
\end{table}

\subsection{Class}
\label{sec:class}

In UML, a Class \citep{uml2.5} is purposed to specify a classification of objects. UML represents atomic classes as named elements of type \textit{Class} without further features. In OWL, the atomic class, \texttt{owl:Class}, has no intension. It can only be interpreted by its name that has a meaning in the world outside the ontology. The atomic class is a class description that is simultaneously a class axiom \citep{kiko2008}. 

\begin{figure}[!ht]
	\centering
	\begin{subfigure}{.5\textwidth}
		\centering
		\begin{tikzpicture}
			\umlsimpleclass{ClassName}
		\end{tikzpicture}		
	\end{subfigure}%
	\begin{subfigure}{.5\textwidth}
		\centering
		\begin{tikzpicture}
			\gClass{}{class}{:ClassName}
		\end{tikzpicture}
	\end{subfigure}
	\caption{Visual representation of a class in UML (on the left) and OWL (on the right)}
	\label{fig:class-visual}
\end{figure}

\begin{trule}[Class -- \coreComponent]
	\label{rule:class-core}
	Specify declaration axiom for UML Class as OWL Class where the URI and a label are deterministically generated from the class name. The label and, if available, the description are ascribed to the class.
\end{trule}

\vspace{-\parskip}
\begin{minipage}[b]{.45\textwidth}
\begin{lstlisting}[language=Turtle, caption={Class declaration in Turtle syntax}, captionpos=b]
:ClassName a owl:Class ;
  rdfs:label "Class name"@en ;
.
\end{lstlisting}
\end{minipage}% <---------------- Note the use of "%"
\quad
\begin{minipage}[b]{.5\textwidth}
\begin{lstlisting}[language=RDF/XML, caption={Class declaration in RDF/XML syntax}, captionpos=b]
<owl:Class rdf:about="http://base.uri/ClassName">  
  <rdfs:label xml:lang="en">Class name</rdfs:label>
</owl:Class>
\end{lstlisting}
\end{minipage}

\begin{trule}[Class -- \shaclComponent]
	\label{rule:class-ds}
	Specify declaration axiom for UML Class as SHACL Node Shape where the URI and a label are deterministically generated from the class name.
\end{trule}

\vspace{-\parskip}
\begin{minipage}[b]{.335\textwidth}
\begin{lstlisting}[language=Turtle, caption={Node shape declaration in Turtle syntax}, captionpos=b]
:ClassName a sh:NodeShape .
\end{lstlisting}
\end{minipage}% <---------------- Note the use of "%"
\quad
\begin{minipage}[b]{.6\textwidth}
%\raggedleft
\begin{lstlisting}[language=RDF/XML, caption={Node shape declaration in RDF/XML syntax}, captionpos=b]
<rdf:Description rdf:about="http://base.uri/ClassName">
<rdf:type rdf:resource="http://www.w3.org/ns/shacl#NodeShape">
<rdf:Description>
\end{lstlisting}
\end{minipage}

\subsection{Abstract class}
\label{sec:class-abstract}

In UML, an abstract Class \citep{uml2.5} cannot have any instances and only its subclasses
can be instantiated.
The abstract classes are declared just like the regular ones (Rule \ref{rule:class-core} and \ref{rule:class-ds}) and in addition a constraint validation rule is generated to ensure that no instance of this class is permitted. 

OWL follows the Open World Assumption \citep{owl2}, therefore, even if the ontology does not contain any instances for a specific class, it is unknown whether the class has any instances. We cannot confirm that the UML abstract class is correctly defined with respect to the OWL domain ontology, but we can detect if it is not using SHACL constraints.

\begin{figure}[!ht]
	\centering
	\begin{subfigure}{.5\textwidth}
		\centering
		\begin{tikzpicture}
			\umlsimpleclass[type=abstract]{ClassName}
		\end{tikzpicture}		
	\end{subfigure}%
	\begin{subfigure}{.5\textwidth}
		\centering
		\begin{tikzpicture}
			\gClass{}{class}{:ClassName}
		\end{tikzpicture}
	\end{subfigure}
	\caption{Visual representation of an abstract class in UML (on the left) and OWL (on the right)}
	\label{fig:class-abstract-visual}
\end{figure}

\begin{trule}[Class -- \shaclComponent]
	\label{rule:class-abstract-ds}
	Specify declaration axiom for UML Class as SHACL Node Shape with a SPARQL constraint that selects all instances of this class.
\end{trule}

\vspace{-\parskip}
\begin{minipage}[b]{.47\textwidth}
\begin{lstlisting}[language=Turtle, caption={Instance checking constraint in Turtle syntax}, captionpos=b]
:ClassName
  rdf:type sh:NodeShape ;
  sh:sparql [
    sh:select """SELECT $this
      WHERE {
        $this a :ClassName .
      }
      """ ;
   ] ;
.
\end{lstlisting}
\end{minipage}% <---------------- Note the use of "%"
\quad
\begin{minipage}[b]{.5\textwidth}
%\raggedleft
\begin{lstlisting}[language=RDF/XML, caption={Instance checking constraint in RDF/XML syntax}, captionpos=b]
<sh:NodeShape rdf:about="http://base.uri/ClassName">
  <sh:sparql rdf:parseType="Resource">  
    <sh:select>SELECT $this
      WHERE {
      $this a :ClassName .
      }
    </sh:select>
  </sh:sparql>
</sh:NodeShape>
\end{lstlisting}
\end{minipage}

\subsection{Attribute}
\label{sec:attribute}

The UML attributes \citep{uml2.5} are properties that are owned by a Classifier, e.g. Class. Both UML attributes and associations are represented by one meta-model element -- Property. OWL also allows one to define properties. A transformation of UML attribute to OWL data property or OWL object property bases on its type. If the type of the attribute is a primitive type it should be transformed into OWL datatype property. However, if the type of the attribute is a structured datatype, class of enumeration , it should be transformed into an OWL object property.

\begin{figure}[!ht]
	\centering
	\begin{subfigure}{.35\textwidth}
		\centering
		\begin{tikzpicture}
			\umlclass[]{ClassName}{}{
			attribute1 : xsd:string \\
			attribute2 : OtherClass \\
			}
		\end{tikzpicture}		
	\end{subfigure}%
	\begin{subfigure}{.65\textwidth}
		\centering
		\begin{tikzpicture}
			\gClass{}{class}{:ClassName}
			\gDatatype{right=8em of class, yshift=2em}{dt}{ xsd:string };
			\gClass{below=2em of dt}{oc}{ :OtherClass } ;
			\gDataProperty{class}{dt}{:attribute1};
			\gObjectProperty{class}{oc}{:attribute2};
		\end{tikzpicture}
	\end{subfigure}
	\caption{Visual representation of class attributes in UML (on the left) and OWL properties (on the right)}
	\label{fig:attribute-visual}
\end{figure}

\begin{trule}[Attribute -- \coreComponent]
	\label{rule:attribute-core}
	Specify declaration axiom(s) for attribute(s) as OWL data or object properties deciding based on their types. The attributes with primary types	should be treated as data properties, whereas those typed with classes or enumerations should be treated as object properties.
\end{trule}

\vspace{-\parskip}
\begin{minipage}[b]{.45\textwidth}
\begin{lstlisting}[language=Turtle, caption={Property declaration in Turtle syntax}, captionpos=b]
:attribute1 a owl:DatatypeProperty ;
  rdfs:label "attribute 1"@en;
  skos:definition "Description of the attribute meaning"@en;
.
:attribute2 a owl:ObjectProperty ;
  rdfs:label "attribute 2"@en;
  skos:definition "Description of the attribute meaning"@en;
.
\end{lstlisting}
\end{minipage}% <---------------- Note the use of "%"
\quad\vspace{-\parskip}
\begin{minipage}[b]{.5\textwidth}
\begin{lstlisting}[language=RDF/XML, caption={Property declaration in  RDF/XML syntax}, captionpos=b]
<owl:DatatypeProperty rdf:about="http://base.uri/attribute1">
  <rdfs:label xml:lang="en">attribute 1</rdfs:label>
  <skos:definition xml:lang="en">Description of the attribute meaning</skos:definition>  
</owl:DatatypeProperty>  
<owl:ObjectProperty rdf:about="http://base.uri/attribute2">
  <rdfs:label xml:lang="en">attribute 1</rdfs:label>
  <skos:definition xml:lang="en">Description of the attribute meaning</skos:definition>  
</owl:ObjectProperty>
\end{lstlisting}
\end{minipage}
\vspace{-\parskip}

\subsection{Attribute owner}

\begin{trule}[Attribute domain -- \reasoningComponent]
	\label{rule:attribute-rc-domain}
	Specify data (or object) property domains for attribute(s).
\end{trule}

\vspace{-\parskip}
\begin{minipage}[b]{.385\textwidth}
\begin{lstlisting}[language=Turtle, caption={Domain specification in Turtle syntax}, captionpos=b]
:attribute1 a owl:DatatypeProperty ;
  rdfs:domain :ClassName ;
.
:attribute2 a owl:ObjectProperty ;
  rdfs:domain :ClassName ;
.
\end{lstlisting}
\end{minipage}% <---------------- Note the use of "%"
\quad
\vspace{-\parskip}
\begin{minipage}[b]{.5\textwidth}
\begin{lstlisting}[language=RDF/XML, caption={Domain specification in  RDF/XML syntax}, captionpos=b]
<owl:DatatypeProperty rdf:about="http://base.uri/attribute1">
  <rdfs:domain rdf:resource="http://base.uri/ClassName"/>
</owl:DatatypeProperty>  
<owl:ObjectProperty rdf:about="http://base.uri/attribute2">
  <rdfs:domain rdf:resource="http://base.uri/ClassName"/>
</owl:ObjectProperty>
\end{lstlisting}
\end{minipage}
\vspace{-\parskip}

\subsection{Attribute type}

\begin{trule}[Attribute type -- \reasoningComponent]
	\label{rule:attribute-rc-range}
	Specify data (or object) property range for attribute(s).
\end{trule}

\vspace{-\parskip}
\begin{minipage}[b]{.384\textwidth}
\begin{lstlisting}[language=Turtle, caption={Range specification in Turtle syntax}, captionpos=b]
:attribute1 a owl:DatatypeProperty;
  rdfs:range xsd:string;
.
:attribute2 a owl:ObjectProperty;
  rdfs:range :OtehrClass;
.
\end{lstlisting}
\end{minipage}% <---------------- Note the use of "%"
\quad
\vspace{-\parskip}
\begin{minipage}[b]{.55\textwidth}
\begin{lstlisting}[language=RDF/XML, caption={Range specification in RDF/XML syntax}, captionpos=b]
<owl:DatatypeProperty rdf:about="http://base.uri/attribute1">
  <rdfs:range rdf:resource="http://www.w3c.org...#string"/>
</owl:DatatypeProperty>  
<owl:ObjectProperty rdf:about="http://base.uri/attribute2">
  <rdfs:range rdf:resource="http://base.uri/OtherClass"/>
</owl:ObjectProperty>
\end{lstlisting}
\end{minipage}
\vspace{-\parskip}

\begin{trule}[Attribute range shape -- \shaclComponent]
	\label{rule:attribute-ds-range}
	Within the SHACL Node Shape corresponding to the UML class, specify property constraints, for each UML attribute, indicating the range class or datatype.
\end{trule}

\vspace{-\parskip}
\begin{minipage}[b]{.37\textwidth}
\begin{lstlisting}[language=Turtle, caption={Property class and datatype constraint in Turtle syntax}, captionpos=b]
:ClassName a sh:NodeShape ;
  sh:property [
    a sh:PropertyShape ;
    sh:path :attribute1 ;
    sh:datatype xsd:string ;
    sh:name "attribute 1" ;
  ];
  sh:property [
    a sh:PropertyShape ;
    sh:path :attribute2 ;
    sh:class :OtherClass ;
    sh:name "attribute 2" ;
  ];
.    
\end{lstlisting}
\end{minipage}% <---------------- Note the use of "%"
\quad\vspace{-\parskip}
\begin{minipage}[b]{.55\textwidth}
\begin{lstlisting}[language=RDF/XML, caption={Property class and datatype constraint in RDF/XML syntax}, captionpos=b]
<sh:NodeShape rdf:about="http://base.uri/ClassName">
<sh:property>
  <sh:PropertyShape>
    <sh:path rdf:resource="http://base.uri/attribute1"/>
    <sh:name>attribute 1</sh:name>
    <sh:datatype rdf:resource="http://www.w3c.org...#string"/>
  </sh:PropertyShape>
</sh:property>
<sh:property>
  <sh:PropertyShape>
    <sh:path rdf:resource="http://base.uri/attribute2"/>
    <sh:name>attribute 2</sh:name>
    <sh:class rdf:resource="http://base.uri/OtherClass"/>
  </sh:PropertyShape>
</sh:property>
</sh:NodeShape>
\end{lstlisting}
\end{minipage}
\vspace{-\parskip}

\subsection{Attribute multiplicity}
\label{sec:attribute-multiplicity}

In \citep{uml2.5}, multiplicity bounds of multiplicity element are specified in the form of
\texttt{[<lower-bound> .. <upper-bound>]}. The lower-bound, also referred here as minimum cardinality or \texttt{min} is of a non-negative
Integer type and the upper-bound, also referred here as maximum cardinality or \texttt{max}, is of an UnlimitedNatural type (see Section \ref{sec:primitive-type}). The strictly
compliant specification of UML in version 2.5 defines only a single value range
for MultiplicityElement. 
%However, in practical examples it is sometimes useful
not limit oneself to a single interval. Therefore, the below UML to OWL
mapping covers a wider case – a possibility of specifying more value ranges for
a multiplicity element. Nevertheless, if the reader would like to strictly follow the
current UML specification, the particular single lower..upper bound interval is
therein also comprised.

\begin{figure}[!ht]
	\centering
	\begin{subfigure}{.35\textwidth}
		\centering
		\begin{tikzpicture}
			\umlclass[]{ClassName}{}{
			attribute1 : xsd:string [2..2]\\
			attribute2 : xsd:string [1..*]\\
			attribute3 : xsd:string [*..2]\\
			attribute4 : xsd:string [1..2]\\
			}
		\end{tikzpicture}		
	\end{subfigure}%
	\begin{subfigure}{.7\textwidth}
		\centering
		\begin{tikzpicture}% it is necessary to escape the square brackets with curly brackets in the TikZ Node text.
			\gClass{}{class}{:ClassName}
			\gGlassRestriction{above=6em of class.west,anchor=west}{oc1}{
			owl:onProperty :attribute1;\\
			owl:cardinality "2";
			};
			\gGlassRestriction{right=5em of class, yshift=2em}{oc2}{
			owl:onProperty :attribute2;\\
			owl:minCardinality "1";
			};
			\gGlassRestriction{right=5em of class, yshift=-2em}{oc3}{
			owl:onProperty :attribute3;\\
			owl:cardinality "2";
			};
			\gGlassRestriction{below=8em of class.west,anchor=west}{oc4}{
			owl:intersectionOf (\\
			{[} owl:onProperty :attribute4;\\
			owl:minCardinality "1"; {]} \\
			{[} owl:onProperty :attribute4;\\
			owl:maxCardinality "2";{]} )
			};

			\gPredicate{class}{oc1}{rdfs:subClassOf};
			\gPredicate{class}{oc2}{...};
			\gPredicate{class}{oc3}{...};
			\gPredicate{class}{oc4}{rdfs:subClassOf};
		\end{tikzpicture}
	\end{subfigure}
	\caption{Visual representation of class attributes with multiplicity in UML (on the left) and OWL class specialising an anonymous restriction of properties (on the right)}
	\label{fig:attribute-multiplicity-visual}
\end{figure}

\begin{trule}[Attribute multiplicity -- \shaclComponent]
	\label{rule:attribute-ds-multiplicity}
	Within the SHACL Node Shape corresponding to the UML class, specify property constraints, corresponding to each attribute, indicating the minimum and maximum cardinality, only where min and max are different from ``*'' (any) and multiplicity is not [1..1]. The expressions are formulated according to the following cases.	
	\begin{enumerate}[label=\Alph*.]
		\item exact cardinality, e.g. [2..2]
		\item minimum cardinality only, e.g. [1..*]
		\item maximum cardinality only, e.g. [*..2]
		\item minimum and maximum cardinality , e.g. [1..2]
	\end{enumerate}
\end{trule}

%attribute 1
\vspace{-\parskip}
\begin{minipage}[b]{.385\textwidth}
\begin{lstlisting}[language=Turtle, caption={Exact cardinality constraint in  Turtle syntax}, captionpos=b]
 :ClassName a sh:NodeShape ;
   sh:property [
       sh:path :attribute1;
       sh:minCount 2 ;
       sh:maxCount 2 ;
       sh:name "attribute 1" ;
     ] ;
 .
\end{lstlisting}
\end{minipage}% <---------------- Note the use of "%"
\quad\vspace{-\parskip}
\begin{minipage}[b]{.6\textwidth}
\begin{lstlisting}[language=RDF/XML, caption={Exact cardinality constraint in RDF/XML syntax}, captionpos=b]
<sh:NodeShape rdf:about="http://base.uri/ClassName">
<sh:property>
  <sh:PropertyShape>
    <sh:path rdf:resource="http://base.uri/attribute1"/>
    <sh:name>attribute 1</sh:name>
    <sh:minCount rdf:datatype="http://www.w3.org...#integer"
        >2</sh:minCount>
    <sh:maxCount rdf:datatype="http://www.w3.org...#integer"
        >2</sh:maxCount>        
  </sh:PropertyShape>
</sh:property>
</sh:NodeShape>
\end{lstlisting}
\end{minipage}
\vspace{-\parskip}
%attribute 2

\vspace{-\parskip}
\begin{minipage}[b]{.385\textwidth}
\begin{lstlisting}[language=Turtle, caption={Min cardinality constraint in Turtle syntax}, captionpos=b]
 :ClassName a sh:NodeShape ;
   sh:property [
       sh:path :attribute2;
       sh:minCount 1 ;
       sh:name "attribute 2" ;
     ] ;
 .
\end{lstlisting}
\end{minipage}% <---------------- Note the use of "%"
\quad\vspace{-\parskip}
\begin{minipage}[b]{.6\textwidth}
\begin{lstlisting}[language=RDF/XML, caption={Min cardinality constraint in RDF/XML syntax}, captionpos=b]
<sh:NodeShape rdf:about="http://base.uri/ClassName">
<sh:property>
  <sh:PropertyShape>
    <sh:path rdf:resource="http://base.uri/attribute2"/>
    <sh:name>attribute 2</sh:name>
    <sh:minCount rdf:datatype="http://www.w3.org...#integer"
        >1</sh:minCount>
  </sh:PropertyShape>
</sh:property>
</sh:NodeShape>
\end{lstlisting}
\end{minipage}
\vspace{-\parskip}

%attribute 3
\vspace{-\parskip}
\begin{minipage}[b]{.385\textwidth}
\begin{lstlisting}[language=Turtle, caption={Max cardinality constraint in Turtle syntax}, captionpos=b]
 :ClassName a sh:NodeShape ;
   sh:property [
       sh:path :attribute3;
       sh:maxCount 2 ;
       sh:name "attribute 3" ;
     ] ;
 .
\end{lstlisting}
\end{minipage}% <---------------- Note the use of "%"
\quad\vspace{-\parskip}
\begin{minipage}[b]{.6\textwidth}
\begin{lstlisting}[language=RDF/XML, caption={Max cardinality constraint in RDF/XML syntax}, captionpos=b]
<sh:NodeShape rdf:about="http://base.uri/ClassName">
<sh:property>
  <sh:PropertyShape>
    <sh:path rdf:resource="http://base.uri/attribute3"/>
    <sh:name>attribute 3</sh:name>
    <sh:maxCount rdf:datatype="http://www.w3.org...#integer"
        >2</sh:maxCount>        
  </sh:PropertyShape>
</sh:property>
</sh:NodeShape>
\end{lstlisting}
\end{minipage}
\vspace{-\parskip}

% attribute 4
\vspace{-\parskip}
\begin{minipage}[b]{.385\textwidth}
\begin{lstlisting}[language=Turtle, caption={Min and max cardinality constraint in Turtle syntax}, captionpos=b]
 :ClassName a sh:NodeShape ;
   sh:property [
       sh:path :attribute4;
       sh:minCount 1 ;
       sh:maxCount 2 ;
       sh:name "attribute 4" ;
     ] ;
 .
\end{lstlisting}
\end{minipage}% <---------------- Note the use of "%"
\quad\vspace{-\parskip}
\begin{minipage}[b]{.6\textwidth}
\begin{lstlisting}[language=RDF/XML, caption={Min and max cardinality constraint in RDF/XML syntax}, captionpos=b]
<sh:NodeShape rdf:about="http://base.uri/ClassName">
<sh:property>
  <sh:PropertyShape>
    <sh:path rdf:resource="http://base.uri/attribute4"/>
    <sh:name>attribute 4</sh:name>
    <sh:minCount rdf:datatype="http://www.w3.org...#integer"
        >1</sh:minCount>
    <sh:maxCount rdf:datatype="http://www.w3.org...#integer"
        >2</sh:maxCount>        
  </sh:PropertyShape>
</sh:property>
</sh:NodeShape>
\end{lstlisting}
\end{minipage}
\vspace{-\parskip}

It should be noted that upper-bound of UML MultiplicityElement can be
specified as unlimited: ``*''. In OWL, cardinality expressions serve to restrict the
number of individuals that are connected by an object property expression to
a given number of instances of a specified class expression \citep{owl2}. Therefore, UML unlimited upper-bound does not add any information to OWL ontology, hence
it is not transformed.

\begin{trule}[Attribute multiplicity -- \reasoningComponent]
	\label{rule:attribute-rc-multiplicity}
	For each attribute multiplicity of the form ( min .. max ), where min and max are different than ``*'' (any), specify a subclass axiom where the OWL class, corresponding to the UML class, specialises an anonymous restriction of properties formulated according to the following cases.	
	\begin{enumerate}[label=\Alph*.]
		\item exact cardinality, e.g. [2..2]
		\item minimum cardinality only, e.g. [1..*]
		\item maximum cardinality only, e.g. [*..2]
		\item maximum and maximum cardinality , e.g. [1..2]
	\end{enumerate}
\end{trule}


%attribute 1
\vspace{-\parskip}
\begin{minipage}[b]{.4\textwidth}
\begin{lstlisting}[language=Turtle, caption={Cardinality restriction in Turtle syntax}, captionpos=b]
:ClassName a owl:Class ;
  rdfs:subClassOf [ a owl:Restriction ;
      owl:cardinality "2"^^xsd:integer;
      owl:onProperty :attribute1 ;
    ] ;
.
\end{lstlisting}
\end{minipage}% <---------------- Note the use of "%"
\quad\vspace{-\parskip}
\begin{minipage}[b]{.6\textwidth}
\begin{lstlisting}[language=RDF/XML, caption={Cardinality restriction in RDF/XML syntax}, captionpos=b]
<owl:Class rdf:about="http://base.uri/ClassName">
  <rdfs:subClassOf>
    <owl:Restriction>
      <owl:onProperty rdf:resource="http://base.uri/attribute1"/>
      <owl:cardinality rdf:datatype="http://www.w3.org...#integer" >2</owl:cardinality>
    </owl:Restriction>
  </rdfs:subClassOf>
</owl:Class>
\end{lstlisting}
\end{minipage}
\vspace{-\parskip}

%attribute 2
\vspace{-\parskip}
\begin{minipage}[b]{.385\textwidth}
\begin{lstlisting}[language=Turtle, caption={Min cardinality restriction in Turtle syntax}, captionpos=b]
:ClassName a owl:Class ;
  rdfs:subClassOf [ a owl:Restriction ;
      owl:minCardinality "1"^^xsd:integer;
      owl:onProperty :attribute2 ;
    ] ;
.
\end{lstlisting}
\end{minipage}% <---------------- Note the use of "%"
\quad\vspace{-\parskip}
\begin{minipage}[b]{.6\textwidth}
\begin{lstlisting}[language=RDF/XML, caption={Min cardinality restriction in RDF/XML syntax}, captionpos=b]
<owl:Class rdf:about="http://base.uri/ClassName">
  <rdfs:subClassOf>
    <owl:Restriction>
      <owl:onProperty rdf:resource="http://base.uri/attribute2"/>
      <owl:minCardinality rdf:datatype="http://www.w3.org...#integer" >1</owl:cardinality>
    </owl:Restriction>
  </rdfs:subClassOf>
</owl:Class>
\end{lstlisting}
\end{minipage}
\vspace{-\parskip}

%attribute 3
\vspace{-\parskip}
\begin{minipage}[b]{.385\textwidth}
\begin{lstlisting}[language=Turtle, caption={Max cardinality restriction in Turtle syntax}, captionpos=b]
:ClassName a owl:Class ;
  rdfs:subClassOf [ a owl:Restriction ;
      owl:maxCardinality "2"^^xsd:integer;
      owl:onProperty :attribute3 ;
    ] ;
.
\end{lstlisting}
\end{minipage}% <---------------- Note the use of "%"
\quad\vspace{-\parskip}
\begin{minipage}[b]{.6\textwidth}
\begin{lstlisting}[language=RDF/XML, caption={Max cardinality restriction in RDF/XML syntax}, captionpos=b]
<owl:Class rdf:about="http://base.uri/ClassName">
  <rdfs:subClassOf>
    <owl:Restriction>
      <owl:onProperty rdf:resource="http://base.uri/attribute3"/>
      <owl:maxCardinality rdf:datatype="http://www.w3.org...#integer" >2</owl:cardinality>
    </owl:Restriction>
  </rdfs:subClassOf>
</owl:Class>
\end{lstlisting}
\end{minipage}
\vspace{-\parskip}

% attribute 4
\vspace{-\parskip}
\begin{minipage}[b]{.385\textwidth}
\begin{lstlisting}[language=Turtle, caption={Min and max cardinality restriction in Turtle syntax}, captionpos=b]
:ClassName a owl:Class ;
  rdfs:subClassOf [
    rdf:type owl:Class ;
    owl:intersectionOf (
      [ a owl:Restriction ;
        owl:minCardinality "1"^^xsd:integer;
        owl:onProperty :attribute4; ]
      [ a owl:Restriction ;
        owl:maxCardinality "2"^^xsd:integer;
        owl:onProperty :attribute4; ]
      ) ;
    ] ;
.        
\end{lstlisting}
\end{minipage}% <---------------- Note the use of "%"
\quad
\vspace{-\parskip}
\begin{minipage}[b]{.6\textwidth}
\begin{lstlisting}[language=RDF/XML, caption={Min and max cardinality restriction in RDF/XML syntax}, captionpos=b]
<owl:Class rdf:about="http://base.uri/ClassName">
  <rdfs:subClassOf>
    <owl:Class>
      <owl:intersectionOf rdf:parseType="Collection">
        <owl:Restriction>
          <owl:onProperty rdf:resource="http://base.uri/attribute4"/>
          <owl:minCardinality rdf:datatype="...#integer"
          >1</owl:minCardinality>
        </owl:Restriction>
        <owl:Restriction>
          <owl:onProperty rdf:resource="http://base.uri/attribute4"/>
          <owl:maxCardinality rdf:datatype="...#integer"
          >2</owl:maxCardinality>
        </owl:Restriction>
      </owl:intersectionOf>
    </owl:Class>
  </rdfs:subClassOf>
</owl:Class>
\end{lstlisting}
\end{minipage}
\vspace{-\parskip}

Attributes with multiplicity exactly one correspond to functional object or data properties in OWL. If we apply the previous rule specifying min and max cardinality will lead to inconsistent ontology. To avoid that it is important that min and max cardinality are not generated from [1..1] multiplicity but only functional property axiom.

\begin{trule}[Attribute multiplicity ``one'' -- \reasoningComponent]
	\label{rule:attribute-rc-multiplicity-one}
	For each attribute that has multiplicity exactly one, i.e. [1..1], specify functional property axiom.
\end{trule}

%attribute 1
\vspace{\parskip}
\begin{minipage}[b]{.384\textwidth}
\begin{lstlisting}[language=Turtle, caption={Declaring a functional property in Turtle syntax}, captionpos=b]
:attribute5 a owl:FunctionalProperty .
\end{lstlisting}
\end{minipage}% <---------------- Note the use of "%"
\quad\vspace{-\parskip}
\begin{minipage}[b]{.6\textwidth}
\begin{lstlisting}[language=RDF/XML, caption={Declaring a functional property in RDF/XML syntax}, captionpos=b]
<rdf:Description rdf:about="http://base.uri/attribute5">
  <rdf:type rdf:resource="http://...owl#FunctionalProperty"/>
</rdf:Description>
\end{lstlisting}
\end{minipage}
\vspace{-\parskip}