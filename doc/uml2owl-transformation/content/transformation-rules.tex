\section{Transformation rules}
\label{sec:tran-rules}


\subsection{Classes}
\label{sec:classes}

UML represents atomic classes as named elements of type \textit{Class} without further features. In OWL, the atomic class, \texttt{owl:Class}, has no intension. It can only be interpreted by its name that has a meaning in the world outside the ontology. The atomic class is a class description that is simultaneously a class axiom. [Kiko2008]



\begin{figure}[!ht]
	\centering
	\begin{subfigure}{.5\textwidth}
		\centering
		\includegraphics[width=.45\linewidth]{logos/EU_Flag_320_213}
		\caption{UML class}
		\label{fig:class-uml}
	\end{subfigure}%
	\begin{subfigure}{.5\textwidth}
		\centering
		\includegraphics[width=.45\linewidth]{logos/EU_Flag_320_213}
		\caption{OWL class}
		\label{fig:class-owl}
	\end{subfigure}
	\caption{Visual representation in UML and OWL of a prototypical class }
	\label{fig:class}
\end{figure}

\begin{lstlisting}[caption={OWL Class},captionpos=b]
Class: C
 skos:prefLabel "label"[en]
 skos:definition "definition"
\end{lstlisting}

\begin{trule}[Class]
	\label{rule:class}
	Given an UML element $E$ of type $Class$ with a name and optionally with a description  create an OWL class $C$. 
	\begin{itemize}
		\item The name is transformed into URI and the label of $C$.
		\item The description of $E$ is transformed into 
	\end{itemize}		 	
\end{trule}

\subsection{Generalisation}
\label{sec:generalisation}
...




