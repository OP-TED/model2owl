%\twocolumn
\section{Transformation of UML connectors}
\label{sec:tran-rules2}

In this section are specified transformation rules for UML association, generalisation and dependency connectors. Table \ref{tab:connectors-overview} provides an overview of the section coverage.

\renewcommand{\thefootnote}{\fnsymbol{footnote}}
%\makesavenoteenv{tabular}
%\makesavenoteenv{table}

% Please add the following required packages to your document preamble:
% \usepackage{booktabs}
\begin{table}[!ht]
\centering
\begin{tabular}{@{}p{0.3\textwidth}p{0.2\textwidth}p{0.2\textwidth}p{0.2\textwidth}@{}}
\toprule
UML element            & Rules \coreComponent & Rules \shaclComponent & Rules \reasoningComponent \\ \midrule
Association\footnotemark[1] & Rule \ref{rule:association-uni-core} &  &  \\
Association domain\footnotemark[1] &  &  & Rule \ref{rule:association-uni-domain-rc} \\
Association range\footnotemark[1] &  & Rule \ref{rule:association-uni-range-dc} & Rule \ref{rule:association-uni-range-rc} \\
Association multiplicity\footnotemark[1] &  & Rule \ref{rule:association-uni-multiplicity-dc} & Rule \ref{rule:association-uni-multiplicity-rc}, \ref{rule:association-uni-multiplicity-one-rc} \\
Association asymmetry\footnotemark[1]  &  & Rule \ref{rule:association-uni-asymetry-ds} & Rule \ref{rule:association-uni-asymetry-rc} \\
Association inverse\footnotemark[2] &  &  & Rule \ref{rule:association-bi-inverse-rc} \\
Dependency\footnotemark[3] & Rule \ref{rule:association-uni-core} & &  \\
Dependency domain\footnotemark[3] &  &  & Rule \ref{rule:association-uni-domain-rc} \\
Dependency range\footnotemark[3] &  & Rule \ref{rule:dependency-uni-range-dc} & Rule \ref{rule:dependency-uni-range-rc} \\
Dependency multiplicity\footnotemark[3] &  & Rule \ref{rule:association-uni-multiplicity-dc} & Rule \ref{rule:association-uni-multiplicity-rc} \\
Class generalisation & Rule \ref{rule:generalisation-class-core} &  &  \\
Property generalisation & Rule \ref{rule:generalisation-property-core} &  &  \\
Class equivalence &  &  & Rule \ref{rule:equivalent-classes-rc} \\
Property equivalence &  &  & Rule \ref{rule:equivalent-properties-rc} \\ \bottomrule
\end{tabular}
\caption{Transformation rules overview for UML connectors}
\label{tab:connectors-overview}
\end{table}

\footnotetext[1]{Applicable to unidirectional and bidirectional connectors}
\footnotetext[2]{Applicable to bidirectional connectors only}
\footnotetext[3]{Inherits all the rules from unidirectional and bidirectional associations}

\renewcommand{\thefootnote}{\arabic{footnote}}

\subsection{Unidirectional association}
\label{sec:association-uni}

A binary Association specifies a semantic relationship between two
member ends represented by properties. Please note that in accordance with
specification \citep{uml2.5}, the association end names are not obligatory. However, we adhere to the UML conventions \citep{costetchi2020b}, where specification of at one member ends, for unidirectional association, and two member ends, for bidirectional association, is mandatory. Moreover, provision of a connector (general) name is discouraged.

\begin{figure}[!ht]
	\centering
	\begin{subfigure}{.385\textwidth}
		\centering
		\begin{tikzpicture}
			\umlsimpleclass{ClassName}
			\umlsimpleclass[below=5em of ClassName]{OtherClass}
%			\umlassoc[geometry=-|-, arg1=tata, mult1=*, pos1=0.3, arg2=toto, mult2=1, pos2=2.9, align2=left]{C}{B}
			\umluniassoc[geometry=--, arg1=relatesTo, mult1=1..*, pos1=.7,   align2=right]{ClassName}{OtherClass}
		\end{tikzpicture}		
	\end{subfigure}%
	\begin{subfigure}{.6\textwidth}
		\centering
		\begin{tikzpicture}
			\gClass{}{class}{:ClassName}
			\gClass{below=5em of class}{oclass}{:OtherClass}
			\gGlassRestriction{right=4.5em of oclass, yshift=2em}{oc2}{
			owl:onProperty :relatesTo;\\
			owl:minCardinality "1";
			};
			\gObjectProperty{class}{oclass}{:relatesTo}
			\gPredicate{class}{oc2}{rdfs:subClassOf}
		\end{tikzpicture}
	\end{subfigure}
	\caption{Visual representation of an UML unidirectional association (on the left) and an OWL property with cardinality restriction on domain class (on the right)}
	\label{fig:association-uni-visual}
\end{figure}

\begin{trule}[Unidirectional association -- \coreComponent]
	\label{rule:association-uni-core}
	Specify object property declaration axiom for the target end of the association. 
\end{trule}

\vspace{-\parskip}
\begin{minipage}[b]{.385\textwidth}
\begin{lstlisting}[language=Turtle, caption={Property declaration in Turtle syntax}, captionpos=b]
:relatesTo a owl:ObjectProperty ;
  rdfs:label "relates oo"@en;
  skos:definition "Description of the relationship meaning"@en;
.
\end{lstlisting}
\end{minipage}% <---------------- Note the use of "%"
\quad\vspace{-\parskip}
\begin{minipage}[b]{.6\textwidth}
\begin{lstlisting}[language=RDF/XML, caption={Property declaration in  RDF/XML syntax}, captionpos=b]
<owl:ObjectProperty rdf:about="http://base.uri/relatesTo">
  <rdfs:label xml:lang="en">relates to</rdfs:label>
  <skos:definition xml:lang="en">Description of the relationship meaning</skos:definition> 
</owl:ObjectProperty>  
\end{lstlisting}
\end{minipage}
\vspace{-\parskip}

\subsection{Association source}

\begin{trule}[Association source -- \reasoningComponent]
	\label{rule:association-uni-domain-rc}
	Specify object property domain for the target end of the association. 
\end{trule}

\vspace{-\parskip}
\begin{minipage}[b]{.385\textwidth}
\begin{lstlisting}[language=Turtle, caption={Domain specification in Turtle syntax}, captionpos=b]
:relatesTo a owl:ObjectProperty ;
  rdfs:domain :ClassName ;
.
\end{lstlisting}
\end{minipage}% <---------------- Note the use of "%"
\quad\vspace{-\parskip}
\begin{minipage}[b]{.6\textwidth}
\begin{lstlisting}[language=RDF/XML, caption={Domain specification in  RDF/XML syntax}, captionpos=b]
<owl:ObjectProperty rdf:about="http://base.uri/relatesTo">
  <rdfs:domain rdf:resource="http://base.uri/ClassName"/>
</owl:ObjectProperty>
\end{lstlisting}
\end{minipage}
\vspace{-\parskip}

\subsection{Association target}

\begin{trule}[Association target -- \reasoningComponent]
	\label{rule:association-uni-range-rc}
	Specify object property range for the target end of the association. 
\end{trule}
\vspace{-\parskip}
\begin{minipage}[b]{.385\textwidth}
\begin{lstlisting}[language=Turtle, caption={Range specification in Turtle syntax}, captionpos=b]
:relatesTo a owl:ObjectProperty ;
  rdfs:range :ClassName ;
.
\end{lstlisting}
\end{minipage}% <---------------- Note the use of "%"
\quad\vspace{-\parskip}
\begin{minipage}[b]{.6\textwidth}
\begin{lstlisting}[language=RDF/XML, caption={Range specification in  RDF/XML syntax}, captionpos=b]
<owl:ObjectProperty rdf:about="http://base.uri/relatesTo">
  <rdfs:range rdf:resource="http://base.uri/ClassName"/>
</owl:ObjectProperty>
\end{lstlisting}
\end{minipage}
\vspace{-\parskip}

\begin{trule}[Association range shape -- \shaclComponent]
	\label{rule:association-uni-range-dc}
	Within the SHACL Node Shape corresponding to the source UML class, specify property constraints indicating the range class.
\end{trule}

\vspace{-\parskip}
\begin{minipage}[b]{.385\textwidth}
\begin{lstlisting}[language=Turtle, caption={Property class constraint in Turtle syntax}, captionpos=b]
:ClassName a sh:NodeShape ;
  sh:property [
    a sh:PropertyShape ;
    sh:path :relatesTo ;
    sh:class :OtherClass ;
    sh:name "relates to" ;
  ];
.    
\end{lstlisting}
\end{minipage}% <---------------- Note the use of "%"
\quad\vspace{-\parskip}
\begin{minipage}[b]{.6\textwidth}
\begin{lstlisting}[language=RDF/XML, caption={Property class constraint in RDF/XML syntax}, captionpos=b]
<sh:NodeShape rdf:about="http://base.uri/ClassName">
<sh:property>
  <sh:PropertyShape>
    <sh:path rdf:resource="http://base.uri/relatesTo"/>
    <sh:name>relates to</sh:name>
    <sh:class rdf:resource="http://base.uri/OtherClass"/>
  </sh:PropertyShape>
</sh:property>
</sh:NodeShape>
\end{lstlisting}
\end{minipage}
\vspace{-\parskip}

\subsection{Association multiplicity}

\begin{trule}[Association multiplicity -- \shaclComponent]
	\label{rule:association-uni-multiplicity-dc}
	Within the SHACL Node Shape corresponding to the source UML class, specify property constraints indicating minimum and maximum cardinality according to cases provided by Rule \ref{rule:attribute-ds-multiplicity}.
\end{trule}

\vspace{-\parskip}\vspace{-\parskip}\vspace{-\parskip}
\begin{minipage}[b]{.385\textwidth}
\begin{lstlisting}[language=Turtle, caption={Min cardinality constraint in Turtle syntax}, captionpos=b]
 :ClassName a sh:NodeShape ;
   sh:property [
       sh:path :relatesTo;
       sh:minCount 1 ;
       sh:name "relates to" ;
     ] ;
 .
\end{lstlisting}
\end{minipage}% <---------------- Note the use of "%"
\quad\vspace{-\parskip}
\begin{minipage}[b]{.6\textwidth}
\begin{lstlisting}[language=RDF/XML, caption={Min cardinality constraint in RDF/XML syntax}, captionpos=b]
<sh:NodeShape rdf:about="http://base.uri/ClassName">
<sh:property>
  <sh:PropertyShape>
    <sh:path rdf:resource="http://base.uri/relatesTo"/>
    <sh:name>relates to</sh:name>
    <sh:minCount rdf:datatype="http://www.w3.org...#integer"
        >1</sh:minCount>
  </sh:PropertyShape>
</sh:property>
</sh:NodeShape>
\end{lstlisting}
\end{minipage}
\vspace{-\parskip}\vspace{-\parskip}\vspace{-\parskip}

\begin{trule}[Association multiplicity -- \reasoningComponent]
	\label{rule:association-uni-multiplicity-rc}
	For the association target multiplicity, where min and max are different than ``*'' (any) and multiplicity is not [1..1], specify a subclass axiom where the source class specialises an anonymous restriction of properties formulated according to cases provided by Rule \ref{rule:attribute-rc-multiplicity}.
\end{trule}

\vspace{-\parskip}
\begin{minipage}[b]{.385\textwidth}
\begin{lstlisting}[language=Turtle, caption={Min cardinality restriction in Turtle syntax}, captionpos=b]
:ClassName a owl:Class ;
  rdfs:subClassOf [ a owl:Restriction ;
      owl:minCardinality "1"^^xsd:integer;
      owl:onProperty :relatesTo ;
    ] ;
.
\end{lstlisting}
\end{minipage}% <---------------- Note the use of "%"
\quad\vspace{-\parskip}
\begin{minipage}[b]{.6\textwidth}
\begin{lstlisting}[language=RDF/XML, caption={Min cardinality restriction in RDF/XML syntax}, captionpos=b]
<owl:Class rdf:about="http://base.uri/ClassName">
  <rdfs:subClassOf>
    <owl:Restriction>
      <owl:onProperty rdf:resource="http://base.uri/relatesTo"/>
      <owl:minCardinality rdf:datatype="http://www.w3.org...#integer" >1</owl:cardinality>
    </owl:Restriction>
  </rdfs:subClassOf>
</owl:Class>
\end{lstlisting}
\end{minipage}
\vspace{-\parskip}

\begin{trule}[Association multiplicity ``one'' -- \reasoningComponent]
	\label{rule:association-uni-multiplicity-one-rc}
	If the association multiplicity is exactly one, i.e. [1..1], specify functional property axiom like in Rule \ref{rule:attribute-rc-multiplicity-one}.
\end{trule}

\vspace{-\parskip}
\begin{minipage}[b]{.385\textwidth}
\begin{lstlisting}[language=Turtle, caption={Declaring a functional property in Turtle syntax}, captionpos=b]
:relatesTo a owl:FunctionalProperty .
\end{lstlisting}
\end{minipage}% <---------------- Note the use of "%"
\quad\vspace{-\parskip}
\begin{minipage}[b]{.57\textwidth}
\begin{lstlisting}[language=RDF/XML, caption={Declaring a functional property in RDF/XML syntax}, captionpos=b]
<rdf:Description rdf:about="http://base.uri/relatesTo">
  <rdf:type rdf:resource="http://...owl#FunctionalProperty"/>
</rdf:Description>
\end{lstlisting}
\end{minipage}
\vspace{-\parskip}

\subsection{Recursive association}
\label{sec:association-self}

\begin{figure}[!ht]
	\centering
	\begin{subfigure}{.4\textwidth}
		\centering
		\begin{tikzpicture}
			\umlsimpleclass{ClassName}
			\umluniassoc[arg=relatesTo, mult=1..*, pos=.9, recursive=-170|-90|7em]{ClassName}{ClassName}
		\end{tikzpicture}		
	\end{subfigure}%
	\begin{subfigure}{.6\textwidth}
		\centering
		\begin{tikzpicture}
			\gClass{}{class}{:ClassName}
			\draw[object-property, ] (class) |- ($(class.south east) + (2em,-2em)$) |- (class) node[midway,fill=white,yshift=-1em,xshift=1em] {:relatesTo};
			\gGlassRestriction{above=3.5em of class}{oc2}{
			owl:onProperty :relatesTo;\\
			owl:minCardinality "1"; };
			\gPredicate{class}{oc2}{rdfs:subClassOf}			
		\end{tikzpicture}
	\end{subfigure}
	\caption{Visual representation of an UML recursive association (on the left) and OWL recursive properties with cardinality restrictions on domain class (on the right)}
	\label{fig:association-self-visual}
\end{figure}

In case of recursive associations, that are from one class to itself, the transformation rules must be applied as in the case of regular unidirectional association, which are from Rule \ref{rule:association-uni-core} to Rule \ref{rule:association-uni-multiplicity-one-rc}. In addition the association must be marked as asymmetric expressed in Rule \ref{rule:association-uni-asymetry-rc}. 

\begin{trule}[Association asymmetry -- \shaclComponent]
	\label{rule:association-uni-asymetry-ds}
	Within the SHACL Node Shape corresponding to the UML class, specify SPARQL constraint selecting instances connected by the object property in a reciprocal manner.
\end{trule}

\vspace{-\parskip}
\begin{minipage}[b]{.385\textwidth}
\begin{lstlisting}[language=Turtle, caption={Declaring an asymmetric property in Turtle syntax}, captionpos=b]
 :ClassName a sh:NodeShape ;
   sh:sparql [
     sh:select """
        SELECT ?this ?that
        WHERE {
        ?this :relatesTo ?that .
        ?that :relatesTo this .
        }""" ; ] ;
 .
\end{lstlisting}
\end{minipage}% <---------------- Note the use of "%"
\quad\vspace{-\parskip}
\begin{minipage}[b]{.55\textwidth}
\begin{lstlisting}[language=RDF/XML, caption={Declaring an asymmetric property in RDF/XML syntax}, captionpos=b]
<sh:NodeShape rdf:about="http://base.uri/ClassName">
  <sh:sparql rdf:parseType="Resource">
    <sh:select>
      SELECT ?this ?that
      WHERE {
      ?this :relatesTo ?that .
      ?that :relatesTo ?this .}
    </sh:select>
  </sh:sparql>
</sh:NodeShape>
\end{lstlisting}
\end{minipage}
\vspace{-\parskip}

\begin{trule}[Association asymmetry -- \reasoningComponent]
	\label{rule:association-uni-asymetry-rc}
	Specify the asymmetry object property axiom for each end of a recursive association.
\end{trule}

\vspace{-\parskip}
\begin{minipage}[b]{.35\textwidth}
\begin{lstlisting}[language=Turtle, caption={Declaring an asymmetric property in Turtle syntax}, captionpos=b]
:relatesTo a owl:AsymmetricProperty .
\end{lstlisting}
\end{minipage}% <---------------- Note the use of "%"
\quad\vspace{-\parskip}
\begin{minipage}[b]{.58\textwidth}
\begin{lstlisting}[language=RDF/XML, caption={Declaring an asymmetric property in RDF/XML syntax}, captionpos=b]
<rdf:Description rdf:about="http://base.uri/relatesTo">
  <rdf:type rdf:resource="http://...owl#AsymmetricProperty"/>
</rdf:Description>
\end{lstlisting}
\end{minipage}
\vspace{-\parskip}

\subsection{Bidirectional association}
\label{sec:association-bi}

\begin{figure}[!ht]
	\centering
	\begin{subfigure}{.21\textwidth}
		\centering
		\begin{tikzpicture}
			\umlsimpleclass{ClassName}
			\umlsimpleclass[below=5em of ClassName]{OtherClass}
			\umlassoc[geometry=-|-, arg1=relatesTo, mult1=1..*, pos1=1.75, arg2=isRelatedTo, mult2=*, pos2=1.25, ]{ClassName}{OtherClass}
%			\umluniassoc[geometry=--, arg1=relatesTo, mult1=1..*, pos1=.7,   align2=right]{ClassName}{OtherClass}
		\end{tikzpicture}		
	\end{subfigure}%
	\begin{subfigure}{.8\textwidth}
		\centering
		\begin{tikzpicture}
			\gClass{}{class}{:ClassName}
			\gClass{below=3.5em of class}{oclass}{:OtherClass}
			\gGlassRestriction{right=8.5em of class}{oc2}{
			owl:onProperty :relatesTo;\\
			owl:minCardinality "1";
			};
			\gObjectProperty{[xshift=.7em]class.south west}{[xshift=.7em]oclass.north west}{:relatesTo}
			\gObjectProperty{[xshift=-.7em]oclass.north east}{[xshift=-.7em]class.south east}{:isRelatedTo}
			\gPredicate{class}{oc2}{rdfs:subClassOf}			
		\end{tikzpicture}
	\end{subfigure}
	\caption{Visual representation of an UML bidirectional association (on the left) and OWL properties with cardinality restrictions on domain class (on the right)}
	\label{fig:association-bi-visual}
\end{figure}

\enlargethispage{2em}
The bidirectional associations should be treated, both on source and target ends, like two unidirectional associations. The transformation rules from Rule \ref{rule:association-uni-core} to Rule \ref{rule:association-uni-multiplicity-one-rc} must be applied to both ends. In addition these rule the inverse relation axiom must be specified. 

\begin{trule}[Association inverse -- \reasoningComponent]
	\label{rule:association-bi-inverse-rc}
	Specify inverse object property between the source and target ends of the association. 
\end{trule}

\vspace{-\parskip}	
\begin{minipage}[b]{.385\textwidth}
\begin{lstlisting}[language=Turtle, caption={Declaring an inverse property in Turtle syntax}, captionpos=b]
:relatesTo owl:inverseOf :isRelatedTo .
\end{lstlisting}
\end{minipage}% <---------------- Note the use of "%"
\quad\vspace{-\parskip}
\begin{minipage}[b]{.6\textwidth}
\begin{lstlisting}[language=RDF/XML, caption={Declaring an inverse property in RDF/XML syntax}, captionpos=b]
<owl:ObjectProperty rdf:about="http://base.uri/relatesTo">
  <owl:inverseOf rdf:resource="http://base.uri/isRelatedTo"/>
</owl:ObjectProperty>
\end{lstlisting}
\end{minipage}
\vspace{-\parskip}

\subsection{Unidirectional dependency}
\label{sec:dependecy}

The UML dependency connectors should be transformed by the rules specified for UML association connectors. 

\subsection{Class generalisation}
\label{sec:generalisation}

Generalisation \citep{uml2.5} defines specialization relationship between Classifiers. In case of UML classes it relates a more specific Class to a more general Class.

UML generalisation set \citep{uml2.5} groups generalizations; incomplete and disjoint constraints indicate that the set is not complete and its specific Classes have no common instances. The UML conventions \citep{costetchi2020b} specify that all sibling classes are by default disjoint, therefore even if no generalisation set is provided it is assumed to be implicit.

\begin{figure}[!ht]
	\centering
	\begin{subfigure}{.5\textwidth}
		\centering
		\begin{tikzpicture}
			\umlsimpleclass{SuperClass}
			\umlsimpleclass[below=3em of SuperClass, xshift=-4em]{ClassName}
			\umlsimpleclass[below=3em of SuperClass, xshift=4em]{OtherClass}
			\umlinherit{ClassName}{SuperClass}
			\umlinherit{OtherClass}{SuperClass}
		\end{tikzpicture}		
	\end{subfigure}%
	\begin{subfigure}{.5\textwidth}
		\centering
		\begin{tikzpicture}
			\gClass{}{superClass}{:SuperClass}
			\gClass{below=4em of superClass, xshift=-4em}{class}{:ClassName}
			\gClass{below=4em of superClass, xshift=4em}{oclass}{:OtherClass}
			\gPredicate{class}{superClass.south west}{rdfs:subClassOf}		
			\gPredicate{oclass}{superClass.south east}{rdfs:subClassOf}
		\end{tikzpicture}
	\end{subfigure}
	\caption{Visual representation of UML generalisation (on the left) and OWL subclass relation (on the right)}
	\label{fig:generalisation-visual}
\end{figure}

\begin{trule}[Class generalisation -- \coreComponent]
	\label{rule:generalisation-class-core}
	Specify subclass axiom for the generalisation between UML classes. Sibling classes must be disjoint with one another. 
\end{trule}

\vspace{-\parskip}
\begin{minipage}[b]{.385\textwidth}
\begin{lstlisting}[language=Turtle, caption={Sub-classification in Turtle syntax}, captionpos=b]
:ClassName rdfs:subClassOf :SuperClass.
:OtherClass rdfs:subClassOf :SuperClass;
   owl:disjointWith :ClassName ;
.
\end{lstlisting}
\end{minipage}% <---------------- Note the use of "%"
\quad\vspace{-\parskip}
\begin{minipage}[b]{.56\textwidth}
\begin{lstlisting}[language=RDF/XML, caption={Sub-classification in RDF/XML syntax}, captionpos=b]
<owl:Class rdf:about="http://base.uri/ClassName">
  <rdfs:subClassOf rdf:resource="http://base.uri/SuperClass"/>
</owl:Class>
<owl:Class rdf:about="http://base.uri/OtherClass">
  <rdfs:subClassOf rdf:resource="http://base.uri/SuperClass"/>
  <owl:disjointWith rdf:resource="http://base.uri/ClassName"/>
</owl:Class>
\end{lstlisting}
\end{minipage}
\vspace{-\parskip}

\subsection{Property generalisation}

Generalization \citep{uml2.5} defines specialization relationship between Classifiers. In case
of the UML associations it relates a more specific Association to more general
Association.

\begin{figure}[!ht]
	\centering
	\begin{subfigure}{.385\textwidth}
		\centering
		\begin{tikzpicture}
			\umlsimpleclass{ClassName}
			\umlsimpleclass[below=5em of ClassName]{OtherClass}
%			\umlassoc[geometry=-|, name=abc, arg1=relatesTo, pos1=1, arg2=isRelatedTo, pos2=.3, anchors=west and west, align1=right, align2=right ]{ClassName}{OtherClass}
			\umlHVHassoc[name=abc, arg1=relatesTo, pos1=1.2, arg2=isRelatedTo, pos2=2.3, anchors=west and west, align1=right, align2=right, arm1=-.5em,arm2=-1em ]{ClassName}{OtherClass}
			\umlHVHassoc[name=def, arg1=hasSister, pos1=1.2, arg2=isSisterOf, pos2=2.3, align1=left, align2=left, anchors=east and east, arm1=.5em,arm2=1em]{ClassName}{OtherClass}
%			\umlassoc[geometry=--, name=def, arg1=hasSister, pos1=1, arg2=isSisterOf, pos2=.3, align1=left, align2=left]{ClassName}{OtherClass}			
			\umlinherit[arm1=2em, arm2=2em]{def-3}{abc-3}
%			\umlVHVinherit[arm1=2em, arm2=2em]{def-1}{abc-1}
%			\draw[data-property] (abc-3) -- (def-1);
		\end{tikzpicture}		
	\end{subfigure}%
	\begin{subfigure}{.7\textwidth}
		\centering
		\begin{tikzpicture}
			\gClass{}{class}{:ClassName}
			\gClass{below=5em of class}{oclass}{:OtherClass}

			\gObjectProperty{[xshift=.7em]class.south west}{[xshift=.7em]oclass.north west}{:relatesTo}
			\gObjectProperty{[xshift=-.7em]oclass.north east}{[xshift=-.7em]class.south east}{:isRelatedTo}
			
			\draw[object-property] (oclass) -- ([shift={(3em,0em)}]oclass.east) -- ([shift={(3em,0em)}]class.east) -- (class) node[rotate=-90,midway,shift={(2.73em,1.5em)},fill=white,](iso) {:isSisterOf};
			
			\draw[object-property] (class) -- ([shift={(-3em,0em)}]class.west) -- ([shift={(-3em,0em)}]oclass.west) -- (oclass) node[,rotate=90,midway,shift={(2.73em,1.5em)},fill=white,](hs) {:hasSister};
			
			\gAxiom{below=2em of oclass.west, xshift=1em}{ax1}{:hasSister rdfs:subPropertyOf\\ :relatesTo .} 	
					
			\gAxiom{below=6em of oclass.east, xshift=-1em}{ax2}{:isSisterOf rdfs:subPropertyOf\\ :isRelatedTo .}
			
			\draw[link] ([xshift=.3em,yshift=-2.7em]hs.south) -- ([xshift=-3em]ax1.north);
			
			\draw[link] ([xshift=.2em,yshift=-4.5em]iso.south) -- ([xshift=3.4em]ax2.north);
		\end{tikzpicture}
	\end{subfigure}
	\caption{Visual representation of UML property generalisation (on the left) and OWL sub-property relation (on the right)}
	\label{fig:generalisation-equal-visual}
\end{figure}

\begin{trule}[Property generalisation -- \coreComponent]
	\label{rule:generalisation-property-core}
	Specify sub-property axiom for the generalisation between UML associations and dependencies.
\end{trule}

\vspace{-\parskip}
\begin{minipage}[b]{.385\textwidth}
\begin{lstlisting}[language=Turtle, caption={Property specialisation in Turtle syntax}, captionpos=b]
:hasSister rdfs:subPropertyOf :relatesTo .
:isSisterOf rdfs:subPropertyOf :isRelatedTo .
\end{lstlisting}
\end{minipage}% <---------------- Note the use of "%"
\quad\vspace{-\parskip}
\begin{minipage}[b]{.56\textwidth}
\begin{lstlisting}[language=RDF/XML, caption={Property specialisation in RDF/XML syntax}, captionpos=b]
<owl:ObjectProperty rdf:about="http://base.uri/hasSister">
  <rdfs:subPropertyOf rdf:resource="http://base.uri/relatesTo"/>
</owl:ObjectProperty>
<owl:ObjectProperty rdf:about="http://base.uri/isSisterOf">
  <rdfs:subPropertyOf rdf:resource="http://base.uri/isRelatedTo"/>
</owl:ObjectProperty>
\end{lstlisting}
\end{minipage}
\vspace{-\parskip}

\subsection{Class equivalence}

\begin{trule}[Equivalent classes -- \reasoningComponent]
	\label{rule:equivalent-classes-rc}
	Specify equivalent class axiom for the generalisation with <<equivalent>> or <<complete>> stereotype between UML classes.
\end{trule}

\enlargethispage{4em}

\vspace{-\parskip}\vspace{-\parskip}
\begin{figure}[!ht]
	\centering
	\begin{subfigure}{.5\textwidth}
		\centering
		\begin{tikzpicture}
			\umlsimpleclass{SuperClass}
			\umlsimpleclass[below=4em of SuperClass]{ClassName}
			\umlinherit[stereo=equivalent]{ClassName}{SuperClass}
		\end{tikzpicture}		
	\end{subfigure}%
	\begin{subfigure}{.5\textwidth}
		\centering
		\begin{tikzpicture}
			\gClass{}{superClass}{:SuperClass}
			\gClass{below=4em of superClass}{class}{:ClassName}
			\gPredicate{class}{superClass.south}{owl:equivalentClass}		
		\end{tikzpicture}
	\end{subfigure}
	\caption{Visual representation of UML class equivalence (on the left) and OWL class equivalence (on the right)}
	\label{fig:generalisation-equivalence-visual}
\end{figure}

\vspace{-\parskip}
\begin{minipage}[b]{.385\textwidth}
\begin{lstlisting}[language=Turtle, caption={Class equivalence in Turtle syntax}, captionpos=b]
:ClassName owl:equivalentClass :SuperClass.
\end{lstlisting}
\end{minipage}% <---------------- Note the use of "%"
\quad\vspace{-\parskip}
\begin{minipage}[b]{.56\textwidth}
\begin{lstlisting}[language=RDF/XML, caption={Class equivalence in RDF/XML syntax}, captionpos=b]
<owl:Class rdf:about="http://base.uri/ClassName">
  <owl:equivalentClass rdf:resource="http://base.uri/SuperClass"/>
</owl:Class>
\end{lstlisting}
\end{minipage}
\vspace{-\parskip}

\subsection{Property equivalence}

\begin{trule}[Equivalent properties -- \reasoningComponent]
	\label{rule:equivalent-properties-rc}
	Specify equivalent property axiom for the generalisation with <<equivalent>> or <<complete>> stereotype between UML properties.
\end{trule}

\vspace{-\parskip}
\begin{minipage}[b]{.385\textwidth}
\begin{lstlisting}[language=Turtle, caption={Property equivalence in Turtle syntax}, captionpos=b]
:hasSister owl:equivalentProperty :relatesTo .
:isSisterOf owl:equivalentProperty :isRelatedTo .
\end{lstlisting}
\end{minipage}% <---------------- Note the use of "%"
\quad\vspace{-\parskip}
\begin{minipage}[b]{.56\textwidth}
\begin{lstlisting}[language=RDF/XML, caption={Property equivalence in RDF/XML syntax}, captionpos=b]
<owl:ObjectProperty rdf:about="http://base.uri/hasSister">
  <owl:equivalentProperty rdf:resource="http://base.uri/relatesTo"/>
</owl:ObjectProperty>
<owl:ObjectProperty rdf:about="http://base.uri/isSisterOf">
  <owl:equivalentProperty rdf:resource="http://base.uri/isRelatedTo"/>
</owl:ObjectProperty>
\end{lstlisting}
\end{minipage}
\vspace{-\parskip}
