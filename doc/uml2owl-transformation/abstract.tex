\section*{Abstract}

	Publications Office of the European Union set off to build an eProcurement ontology. The ultimate objective of the project is to put forth a commonly agreed ontology that will conceptualise, formally encode and make available in an open, structured and machine-readable format data about public procurement, covering end-to-end procurement, i.e. from notification, through tendering to awarding, ordering, invoicing and payment.
	
	The process and the methodology adopted involve modelling the conceptual model in Unified Modelling Language (UML) and then, by abiding a set of conventions and recommendations, transform that model into a formal ontology expressed in \mbox{Web Ontology Language (OWL)}. 
	
	This document provides a working definition of the transformation rules from the UML conceptual model into the formal OWL ontology and validation data shapes. These rules are organised in accordance with the eProcurement ontology \mbox{architecture}.