\section*{Abstract}
	Public procurement is undergoing a digital transformation. The EU supports the process of rethinking public procurement process with digital technologies in mind. This goes beyond simply moving to electronic tools; it rethinks various pre-award and post-award phases. The aim is to make them simpler for businesses to participate in and for the public sector to manage. It also allows for the integration of data-based approaches at various stages of the procurement process.
	
	With digital tools, public spending should become more transparent, evidence-oriented, optimised, streamlined and integrated with market conditions. This puts eProcurement at the heart of other changes introduced to public procurement in new EU directives.
	
	Given the increasing importance of data standards for eProcurement, a number of initiatives driven by the public sector, the industry and academia have been kick started	in the recent years. Some have grown organically, while others are the result of	standardisation work.
	
	In this context, the Publications Office of the European Union aims to develop an eProcurement ontology.
	
	The objective of the eProcurement ontology is to act as this common standard on the	conceptual level, based on consensus of the main stakeholders and designed to encompass the major requirements of the eProcurement process in conformance with the Directives and Regulations.
	
	This document provides a working definition of what is the architectural stance and the design decisions that shall be adopted for the eProcurement ontology along with the specifications how to generate it. 