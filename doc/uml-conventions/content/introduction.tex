\section{Introduction}
\label{sec:introduction}

%TODO: rephrase section; add references

The conceptual model is represented as a UML class diagram. This diagram comprises primarily packages, classes, data types, enumerations, enumeration items, class attributes, association relation and dependency relation. There are additional elements which will be addressed contextually below. 

The UML Conceptual Model of the eProcurement domain represents a valuable resource for generation of the formal eProcurement ontology. It is, in principle, possible to generate automatically the formal ontology from the UML model, provided that a set of clear transformation rules are established, and that a set of modelling conventions is respected. 

This document aims at establishing the conventions for and assumptions about the UML (v.2.5) model such that it can be further automatically transformed from its XMI (v.2.5.1) serialisation into a formal ontology in RDF format [ref rdf].  This process is described elsewhere [ref] and the main concern here is defining constraints and recommendations for the  UML conceptual model. 

\subsection{Key words for Requirement Statements}
\label{sec:keywords}
The key words ``MUST'', ``MUST NOT'', ``REQUIRED'', ``SHALL'', ``SHALL  NOT'', ``SHOULD'', ``SHOULD NOT'', ``RECOMMENDED'',  ``MAY'', and ``OPTIONAL'' in this document are to be interpreted as described in RFC 2119 \cite{rfc2119}.

The key words ``MUST (BUT WE KNOW YOU WON'T)'', ``SHOULD CONSIDER'', ``REALLY SHOULD NOT'', ``OUGHT TO'', ``WOULD PROBABLY'', ``MAY WISH TO'', ``COULD'', ``POSSIBLE'', and ``MIGHT'' in this document are to be interpreted as described in RFC 6919 \cite{rfc6919}.

\subsection{Context}
\label{sec:context}
