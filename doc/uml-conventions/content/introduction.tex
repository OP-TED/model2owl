\section{Introduction}
\label{sec:introduction}
	
	This document provides a working specification of the guidelines and conventions for the eProcurement conceptual model, such that it qualifies as suitable input for the transformation scripts meant to generate the formal eProcurement ontology. 
	
	The business context and the project overview are detailed in \citet{costetchi2020a}, while the next section provides the motivation and situates the current document within the project.
	
	\subsection{Context}
	
	In the eProcurement ontology project, the conceptual model was decided \cite{d2.01-2017} to be represented in \textit{Unified Modelling Language (UML)} \citep{uml-userguide}. It is a visual representation language that facilitates understanding and convergence between stakeholders towards a common conceptualisation of the model. 
	
	Generally, the primary application of UML \citep{fowler2004} for ontology design is in the specification of class diagrams initially conceived for object-oriented software. UML does not define a formal semantics that would permit to determine, from the class diagrams, whether an ontology is consistent, or to determine the correctness of the ontology implementation. Semantics in such cases becomes a subject to interpretation by  the stakeholders involved in the development process and later by the users in the application and integration tasks \cite{grunninger2003}.
	
	On the other hand, UML is closer than more logic-oriented approaches to the programming languages in which enterprise applications are implemented. For this reason the current specification establishes conventions for interpretation of the UML-based conceptual model.
		
	The UML Conceptual Model of the eProcurement domain serves as the single source of truth, which means that the formal eProcurement ontology is derived from it through a model transformation process. It is possible to generate automatically the formal ontology in RDF format \cite{rdf11} from the XMI (v.2.5.1) serialisation \cite{xmi2.5.1} of an UML (v.2.5) model \cite{uml2.5}, provided that a set of clear transformation rules are established \citep{costetchi2020c}, and that a set of modelling conventions is respected.
				
	This document provides the UML modelling constraints and a set of conventional and technical recommendations for naming and structuring the UML class diagram elements: packages, classes, data types, enumerations, enumeration items, class attributes, association relation and dependency relation. There are additional elements which will be addressed contextually in the following sections. 

	\subsection{Requirements}
	\label{sec:requirements}
	
	The eProcurement conceptual model must fulfil mainly four fundamental objectives.
	\begin{itemize}
		\item Facilitate understanding/comprehension of the represented system
		\item Promote efficient conveyance of system details between team members and external stakeholders.
		\item Provide a point of reference for system designers to gather system specifications and documentation.
		\item Serve as input for development of a (more) formal model.		
	\end{itemize}
	
	In order to support objectives a conceptual model should fulfil the following requirements.
	\begin{itemize}
		\item Be available to all team members, to facilitate collaboration and iteration.
		\item Be easily changeable, as a continuous reflection of up-to-date information.
		\item Contain both visual and written forms of diagramming, to better explain the abstract concepts it may represent.
		\item Establish relevant terms and concepts that will be used throughout the project.
		\item Define said terms and concepts.
		\item Provide a basic structure for entities of the project.
		\item Reduce the ambiguity while maintaining a simple and concise encoding.  
	\end{itemize}
	
	\subsection{Key words for Requirement Statements}
	\label{sec:keywords}
	
	The key words ``MUST'', ``MUST NOT'', ``REQUIRED'', ``SHALL'', ``SHALL  NOT'', ``SHOULD'', ``SHOULD NOT'', ``RECOMMENDED'',  ``MAY'', and ``OPTIONAL'' in this document are to be interpreted as described in RFC 2119 \cite{rfc2119}.
	
	The key words ``MUST (BUT WE KNOW YOU WON'T)'', ``SHOULD CONSIDER'', ``REALLY SHOULD NOT'', ``OUGHT TO'', ``WOULD PROBABLY'', ``MAY WISH TO'', ``COULD'', ``POSSIBLE'', and ``MIGHT'' in this document are to be interpreted as described in RFC 6919 \cite{rfc6919}.